%Bien démarrer un document:
\documentclass[ (options) ]{ (classe) }
\usepackage[utf8]{inputenc}
\useackage[T1]{fontenc}
% Autres packages nécessaires.
\usepackage[french]{babel}

% Commandes personnalisées.

\title{ (titre) }
\author{ (auteur) }
\date{ (date) }

\begin{document}
\maketitle

\end{document}

% Les Options applicables à la commande \documentclass{} :
+ Format du papier : a4paper, a5paper, letterpaper, b5paper…
  Default: letterpaper
+ Taille de la police principale : 10pt, 11pt, 12pt
  Default: 10pt
+ Alignement des équations : fleqn (à gauche, centrées par défaut)
+ Colonnes : onecolumn, twocolumn
  Default: onecolumn
+ Première page des chapitres : openany, openright
  Default: openright
+ Recto verso : oneside, twoside
  Default: article et report : oneside
           book : twoside

% Les accents accessibles sous LaTeX sont les suivants :
+ \`{a} ou \`a accent grave
+ \'{e} ou \'e accent aigu
+ \^{i} ou \^i accent circonflexe
+ \"{o} ou \"o trema
+ \~{u} ou \~u tilde
+ \={o} ou \=o surligné
+ \.{o} ou \.o point
+ \u{o}
+ \v{o}
+ \H{o} trema hongrois
+ \t{oo}
+ \c{c} cédille
+ \d{o} point en dessous
+ \b{o} sousligné

% Divers/Astuces
+ Pas de \chapter{} avec les \documentclass{article} % Pas de chapitre avec les document de type "article"
+ Aller à la ligne sans créer de nouveau paragraphe : \\ ou \newline
+ Aller à la ligne et créer un nouveau paragraphe : 2 sauts de lignes
+ Saut de page : \newpage
+ Pour modifier les marges : \usepackage[top=2cm, bottom=2cm, left=2cm, right=2cm]{geometry}



