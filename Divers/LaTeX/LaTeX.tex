%Bien démarrer un document:
\documentclass[ (options) ]{ (classe) }
\usepackage[utf8]{inputenc}
\useackage[T1]{fontenc}
% Autres packages nécessaires.
\usepackage[french]{babel}

% Commandes personnalisées.

\title{ (titre) }
\author{ (auteur) }
\date{ (date) }

\begin{document}
\maketitle

\end{document}

% Les Options applicables à la commande \documentclass{} :
+ Format du papier : a4paper, a5paper, letterpaper, b5paper…
  Default: letterpaper
+ Taille de la police principale : 10pt, 11pt, 12pt
  Default: 10pt
+ Alignement des équations : fleqn (à gauche, centrées par défaut)
+ Colonnes : onecolumn, twocolumn
  Default: onecolumn
+ Première page des chapitres : openany, openright
  Default: openright
+ Recto verso : oneside, twoside
  Default: article et report : oneside
           book : twoside

% Les accents accessibles sous LaTeX sont les suivants :
+ \`{a} ou \`a accent grave
+ \'{e} ou \'e accent aigu
+ \^{i} ou \^i accent circonflexe
+ \"{o} ou \"o trema
+ \~{u} ou \~u tilde
+ \={o} ou \=o surligné
+ \.{o} ou \.o point
+ \u{o}
+ \v{o}
+ \H{o} trema hongrois
+ \t{oo}
+ \c{c} cédille
+ \d{o} point en dessous
+ \b{o} sousligné

% Divers/Astuces
+ Pas de \chapter{} avec les \documentclass{article} % Pas de chapitre avec les document de type "article"
+ Aller à la ligne sans créer de nouveau paragraphe : \\ ou \newline
+ Aller à la ligne et créer un nouveau paragraphe : 2 sauts de lignes
+ Saut de page : \newpage
+ Pour modifier les marges : \usepackage[top=2cm, bottom=2cm, left=2cm, right=2cm]{geometry}

% Listes à puces:
\begin{itemize}
\item Un canard.
\item Un mammouth.
\item[@] Une pintade. % En plaçant un @ entre crochets après \item, transforme la puce en @
\item[0] Un lapin.
\end{itemize}

% Listes numérotées:
\begin{enumerate}
\item un canard
\item un mammouth
\item[@] une pintade % En plaçant un @ entre crochets après \item, transforme la puce en @
\item[0] un lapin
\end{enumerate}

% Listes de description:
\begin{description}

\item[Un canard :] bestiole qui fait coin.
\item[Un poulpe :] bestiole qui fait bloub.
\item[Un ornithorynque :] bestiole qui fait rire.
\item[Un ours :] bestiole qui fait mal.

\end{description}

% Les styles:
\pagestyle{plain} % classique: numéro de page en pied de page, centré
\pagestyle{headings} % type book: chapitre et numéro de page en tête de page 
\pagestyle{empty} % rien

% Les tailles de texte:
\tiny Minuscule
\scriptsize Très très petite
\footnotesize Très petite
\small Petite
\normalsize Normale (définie dans \documentclass)
\large Légèrement plus grande que la normale
\Large Grande
\LARGE Très grande
\huge Très très grande
\Huge Énorme !

% Notes de bas de page
- text \footnote{Texte de la note.} text % Numérotation automatique
- text /footnotemark[x] % Numérotation manuelle
  \footnotetext[x]{Texte de la note.}

% LaTeX permet d'écrire des références internes de façon simple. 
Pour ce faire, trois commandes sont à connaître. 
La commande \label{nom_choisi} sert à marquer un endroit, et les commandes \ref{nom_choisi} 
et {\pageref{nom_choisi}} vous permettent d'appeler le numéro de page 
ou la référence de l'élément marqué dans une autre zone de votre document

